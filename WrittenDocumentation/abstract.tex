\thispagestyle{plain}
\begin{center}
    \Large
    \textbf{Abstract}
\end{center}

Stock market has been a studied domain for a long time especially due to the
attention received from investors. More specifically, stock market prediction is known for its complexity and volatility. Initially, the task was done solely by humans, but lately with the technological evolution more and more operations are automated including the actual prediction. In the beginning of artificial intelligence, basic methods were used to complement statistical methods. Moving forward through the timeline more methods were tested and as with most domains in which artificial intelligence was used the current standard involves more or less machine learning. 

This thesis will approach the stock market prediction problem as a time series forecasting one. It will be seen from the two classic perspectives: as a classification problem predicting only a general trend such as increase, decline or stagnation and as a regression problem predicting a set of prices during different time horizons. The problem will be solved from a deep learning perspective experimenting with different architectures which will mainly consist of specific variations of recurrent neural networks which have proven their effectiveness in other problems from other domains: long short-term memory networks and gated recurrent unit networks.

In terms of results, ... TODO

This work is the result of my own activity. I have neither given nor received unauthorized assistance on this work.

\vfill
Copyright © 2020, Flaviu-Andrei Jurj

All rights reserved.

The author hereby grants to Babeş-Bolyai University Cluj-Napoca permission to reproduce and to distribute publicly paper and electronic copies of this thesis document in whole or in part.
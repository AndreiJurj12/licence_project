The stock market can be defined as a loose network of economic transactions of stocks (also called shares) which represent ownership claims on businesses. The act of investing has changed a lot during recent years becoming more and more automated through various electronic platforms which even try to predict the actual movement of the market.

There are many strategies which may be adopted by an investor, but ultimately they reduce to the most important task: the capability of predicting to some degree the movement of the market. In this thesis, we will evaluate the reliability of machine learning for tackling this problem. Technical analysis seeks to determine the future price of a stock based only on the previous trends of the price without taking into account details and fundamentals of the company.

This thesis will present a generic perspective regarding the problem of stock market prediction. A section concerning the recent techniques and related work will drive some of the information for the future problem solving. A comparison will be attached regarding the possible variations and various architectures will be compared. Finally, an in-depth presentation of the results will be followed by the lifecycle of developing an application which will incorporate the previous results, going through each stage and phase necessary.

My solution and results ... TODO

In the following chapters, a complete perspective will be given to stock market prediction. The thesis starts with a serious of subsections regarding the stock market fundamentals, difficulties and controversy. This section will build the necessary information while responding to some of the known arguments regarding stock market prediction. Further on, the scientific problem will be presented with its variations (classification and regression). The methodology regarding the used and preprocessed training data will be complemented by the proposed approach and the artificial intelligence concepts behind it. Finally, an in-depth display and comparison will be done regarding the obtained results and issues. Moving on, the next chapter will present some key concepts with respect to the technologies used in order to develop the application. The following set of sections will present the applied stages in the lifecycle development process starting with stakeholder requirements and advancing through various design stages reaching the implementation and end goal - the user manual. Ultimately, a small chapter of conclusions will be presented summarizing the contributions and results obtained, ending with suggested further improvements.